% Latex template
\documentclass[11pt,a4paper]{amsart}
\usepackage{amsmath,amsfonts,amsthm,amssymb,color,dsfont,tikz-cd}

\usepackage{hyperref}
\usepackage[alphabetic,initials]{amsrefs}

%%%%%%%%%%%%%%%%%%%%%%%%% preamble

% theorem-type environments
\theoremstyle{plain}
\newtheorem{prop}{Proposition}[section]
\newtheorem{thm}[prop]{Theorem}
\newtheorem{cor}[prop]{Corollary}
\newtheorem{lemma}[prop]{Lemma}

\theoremstyle{definition}
\newtheorem{example}[prop]{Example}
\newtheorem{defn}[prop]{Definition}

\theoremstyle{remark}
\newtheorem{remark}[prop]{Remark}

\numberwithin{equation}{section}

% user-defomed macros
\newcommand{\defin}{\textbf}
\newcommand{\CC}{{\mathbb C}}
\newcommand{\cov}{{\operatorname{cov}}}
\newcommand{\eE}{{\mathcal E}}
\newcommand{\NN}{{\mathbb N}}
\newcommand{\PP}{{\mathbb P}}
\newcommand{\ZZ}{{\mathbb Z}}
\renewcommand{\SS}{{\mathbb S}}
\newcommand{\DD}{{\mathbb D}}
\newcommand{\RR}{{\mathbb R}}
\newcommand{\QQ}{{\mathbb Q}}
\newcommand{\rR}{{\mathcal R}}
\newcommand{\OO}{{\mathcal O}}
\newcommand{\p}{\partial}
\newcommand{\mM}{{\mathcal M}}
\newcommand{\pP}{{\mathcal P}}
\newcommand{\iI}{{\mathcal I}}
\newcommand{\jJ}{{\mathcal J}}
\newcommand{\uU}{{\mathcal U}}
\newcommand{\sS}{{\mathfrak S}}
\newcommand{\1}{{\mathds 1}}
\newcommand{\Crit}{\operatorname{Crit}}
% colours
\definecolor{red}{rgb}{1,0,0}
\newcommand{\red}[1]{{\color{red}#1}}
\newcommand{\GKK}{{G_{\bar{K} : K}}}
\newcommand{\st}{{\text{s.t.}}}
\newcommand{\ra}{\rightarrow}
\newcommand{\Sel}{\text{Sel}}
\newcommand{\TS}{\text{TS}}

\title[Galois Cohomology]{Galois Cohomology}

\begin{document}

\maketitle

\section{Group Cohmology}

Let $G$ be a finite group. We say an abelian group $A$ is a $G$-module if $A$
is a module for the group ring $\ZZ[G] = \{ \sum_{i=1}^n x_i g_i |
x_i \in \ZZ , g_i \in G , n \, \text{finite}\}$. We define the
first and second cohomology groups of $A$ by
\[H^0 (G, A) = \text{Hom}_{\ZZ[G]}(\ZZ, A)\] and
\[H^1 (G, A) = \text{Ext}^1_{\ZZ[G]} (\ZZ, A)\]
where $\ZZ$ is considered the trivial $G$-module where $g x = x$ for every
$x \in \ZZ, g \in G$.
Let $\phi : \ZZ \rightarrow A$ be a module homomorphism. Then
\[\begin{split} \phi(1) &= \phi(g \cdot 1) \,\,\,\, \forall g \in G \\
    &= g \cdot \phi(1) , \end{split}\]
so $\phi(1) \in A^G = \{x \in A \, | \, gx = x, \forall g \in G\}$, the set
of elements at which $G$ acts trivially. Since a $\ZZ$ homomorphism is
completely determined by the image of 1, we can set
\[ H^0 (G, A) = \text{Hom}_{\ZZ[G]}(\ZZ, A) = A^G .\] 

We can also define the first cohomology group as
\[H^1(G, A) = \frac{Z^1(G,A)}{B^1(G,A)}\]
where
\[Z^1(G,A) = \{ \phi : G \rightarrow A \, | \phi(gh) = \phi(g) + g \phi(h)\}\]
\[B^1(G,A) = \{ \delta \in Z^1 \, | \, \exists a \in A \,\,\text{such that}\,\, \delta(g)
  = ga - a , \forall g \in G\}\]

\begin{prop}
  With this setup, we can take a short exact sequence of $G$-modules
  \[ 0 \rightarrow A \rightarrow B \rightarrow C \rightarrow 0\]
  and form the long exact sequence
  \[ 0 \rightarrow H^0(G, A) \rightarrow H^0(G,B) \rightarrow H^0(G,C)
    \rightarrow H^1(G, A) \rightarrow H^1(G,B) \rightarrow H^1(G,C) \]
\end{prop}

\section{Galois Cohomology}

We can now use the machinery of group cohomology to study number fields.
Let $L, K$ be number fields and $L : K$ a finite degree Galois
extension with Galois group $G_{L:K}$.
A Galois module $A$ is a module over $G_{L:K}$.

\begin{example}
  We can consider the Galois modules $A = L \simeq K[G], $ or $A =
  L^{\times}$ where $L^{\times}$ is the multiplicative group of $L$.
\end{example}

\begin{example}
  If $E$ is an elliptic curve over $K$ then $A = E(L)$ is a Galois module,
  since the addition formulas are rational in $\QQ$.
\end{example}

\begin{defn}
  The first Galois cohomology group is $H^1(L:K, A) = H^1(G_{L:K}, A)$.
\end{defn}

The following result gives an important terminating condition for the
long exact sequences:

\begin{thm}[Hilbert's Theorem 90]
  $  H^1(L : K, L^{\times}) = 0$
\end{thm}
Proof: Serre \\

In what follows we will take $L = \bar{K}$, the algebraic closure of K and
$G_K = G_{\bar{K}:K}$ its Galois group. This extension is usually infinite, so
it will be necessary to make amendmends to the previous defintions.

\begin{defn}
  A $G_K$-module $A$ is called a continuous $G_K$-module if for all $g \in G_K, a \in A$,
  there exists a finite Galois extension $L:K$ such that $g(a)$
  depends only on the image of $g$ in $G_{L:K}$.
\end{defn}

\begin{example}
  $\bar{K}, \bar{K}^{\times}$ and $E(\bar{K})$ are all continuous $G_K$-modules.
\end{example}

To form homology groups, set
\[H^1(K,A) = \frac{Z^1_{\text{cts}}(K,A)}{B^1(K,A)}\]
where
\[Z^1_{\text{cts}} = \{\phi : G_K \rightarrow A \, | \, \exists L : K
  \text{ such that } \phi(g) \text{ depends only on } g / L\}.\]

If we have a short exact sequence of continuous $G_K$-modules
\[ 0 \ra A \ra B \ra C \ra 0\]
this gives a long exact sequence of Galois cohomology groups
\[0 -> A^{G_K} \ra B^{G_K} \ra C^{G_K} \ra H^1(K,A) \ra H^1(K,B) \ra H^1(K,C).\]

\section{Applications to Elliptic Curves}
A crucial step in the proof of the Mordell-Weil theorem, enough to show that
the rank of an elliptic curve over $\QQ$ is finite, is the study of the size
of the quotient $E(\QQ) / 2 E(\QQ)$. In curves with at least one 2-torsion
point, this can be done with the help of an isogeny.

Let $$E : y^2 = x^3 + ax^2 + bx$$ be an elliptic curve over $\QQ$. We know $E$
has the rational points $\OO, T = (0,0)$.

We also consider the curve
\[ \bar{E} : y^2 = x^3 + \bar{a}x^2 + \bar{b}x \]
where $\bar{a} = -2a$ and $\bar{b} = a^2 - 4b$. Repeating this process yields
the curve
\[ \bar{\bar{E}} : y^2 = x^3 +  4ax^2 + 16bx\]
which is birationally equivalent to $E$ by the transformation $y \mapsto 8y,
x \mapsto 4x$.

\begin{prop}
  Let $E, \bar{E}$ be as above. The maps $\phi : E \rightarrow \bar{E}$ and
  $\psi : \bar{E} \rightarrow E$ defined by
  \[\phi(P) =
    \begin{cases}
      (\frac{y^2}{x^2}, \frac{y(x^2-b)}{x^2}), & \text{ if } P \neq \OO, T \\
      \bar{\OO}, & \text{ if } P = \OO \text{ or } P = T
    \end{cases}\]
  and
  \[\psi(P) =
    \begin{cases}
      (\frac{\bar{y}^2}{4\bar{x}^2}, \frac{\bar{y}(\bar{x}^2-\bar{b})}{8\bar{x}^2}),
      & \text{ if } P \neq \bar{\OO}, \bar{T} \\
      \bar{\OO}, & \text{ if } \bar{P} = \bar{\OO} \text{ or } \bar{P} = \bar{T}
    \end{cases}\]
  are elliptic curve homomorphisms, $\text{Ker}(\phi) = \{\OO, T\}$ and
  \[ \psi \circ \phi (P) = 2P , \text{ for all points } P \in E.\]
\end{prop}
Proof: Silverman and Tate

\begin{lemma}
  If $E / \psi(\bar{E}))$ and $\bar{E} / \phi(E)$ are finite, then so is $E/2E$
\end{lemma}
Proof: Silverman and Tate. 

We are thus led to consider the quotient $E(\QQ) / \psi(\bar{E}(\QQ))$ (the
other one can be treated identically).

By the proposition, we have a short exact sequence of $G_{\QQ}$-modules
\[ 0 \rightarrow \{\OO, T\} \rightarrow \bar{E}(\bar{\QQ}) \xrightarrow[]{\psi} E(\QQ)
  \rightarrow 0\]
and $\{\OO, T\} \simeq \ZZ/2\ZZ$. Taking Galois cohomology we get the long
exact sequence

\[ 0 \rightarrow \ZZ / 2\ZZ \rightarrow \bar{E}(\QQ) \xrightarrow[]{\psi} E(\QQ)
  \rightarrow H^1(\QQ, \ZZ/2\ZZ) \rightarrow H^1(\QQ, \bar{E}) \xrightarrow{H^1(\psi)}
  H^1(\QQ, E)\]
This in turn gives us the short exact sequence
\[ 0 \rightarrow \frac{E(\QQ)}{\psi(\bar{E} (\QQ))} \rightarrow
  H^1(\QQ, \ZZ / 2\ZZ) \rightarrow H^1(\QQ, \bar{E}(\QQ))[\psi] \rightarrow 0\]

where $H^1(\QQ,\bar{E}(\QQ))[\psi] = (H^1(\psi))^{-1}.$ \vspace{3mm}

To get the order of $E(\QQ) / \psi(\bar{E}(\QQ))$ we need to know the order of
$H^1(\QQ, \ZZ/2\ZZ)$.

\begin{prop}
  $H^1(\QQ, \ZZ/2\ZZ) \simeq \QQ^{\times} / (\QQ^{\times})^2$
\end{prop}
Proof: Consider the exact sequence of Galois modules
\[0 \rightarrow \mu_2 \rightarrow \bar{\QQ}^{\times} \xrightarrow[]{2} \bar{\QQ}^{\times} \ra
  \, 0 \]
where $\mu_2 = \text{Gal} (\QQ(i) : \QQ) \simeq \ZZ / 2\ZZ.$

Taking cohomology gives the long exact sequence
\[ 0 \ra \mu_2 \ra \bar{\QQ}^{\times} \xrightarrow[]{2} \bar{\QQ}^{\times} \ra H^1(\QQ, \ZZ/2\ZZ)
  \ra H^1(\QQ, \bar{\QQ}^{\times})\]
and $H^1(\QQ, \bar{\QQ}^{\times}) \simeq 0$ by Hilbert's Theorem 90. Thus,
\[H^1(\QQ, \ZZ/2\ZZ) \simeq \QQ^{\times}/(\QQ^{\times})^2. \]

\section{The Selmer and Tate-Shafarevich Groups}

In the effort to understand $E/2E$, we were led to consider the quotient
$E(\QQ) / \psi(E(\QQ))$. Our application of Galois cohomology showed that this
group is closed related to the multiplicative group of rationals modulo squares
$\QQ^{\times} / (\QQ^{\times})^2$. This group, in turn, can be studied by means
of local method.

\begin{defn}
  A place $\nu$ is either a prime number $p$ or $\infty$. $\QQ_\nu$ then is
  either the field of p-adic numbers if $\nu = p$ or $\RR$ if $\nu = \infty$.
\end{defn}
There is then a natural inclusion $\QQ \ra \prod\limits_\nu \QQ_\nu,
x \mapsto (x \mod 2, x \mod 3, \dots, x).$ This extends to the commutative diagram

\begin{tikzcd}
  0 \arrow[r] &
  E(\QQ)/\psi(E(\QQ)) \arrow[r] &
  \QQ^{\times}/(\QQ^{\times})^2 \arrow[r] \arrow[d] &
  H^1(\QQ, \bar{E})[\psi] \arrow[r] \arrow[d] &
  0 \\
  0 \arrow[r] &
  \prod\limits_{\nu} E(\QQ_\nu) / \psi(E(\QQ_\nu)) \arrow[r] &
  \prod\limits_\nu(\QQ_\nu^{\times}/(\QQ_\nu)^{\times})^2 \arrow[r] &
  \prod\limits_\nu H^1(\QQ_\nu, \bar{E})[\psi] \arrow[r] &
  0
\end{tikzcd}

We are now ready to make the following definition

\begin{defn}[Selmer Group]
  The Selmer group of the elliptic curve $E$ is defined
  $$\Sel(E) = \text{Ker}\big((\QQ^{\times}/(\QQ^{\times})^2 \ra
  \prod\limits_\nu H^1(\QQ_\nu, \bar{E})[\psi]\big)$$
\end{defn}

This group is important in rank computations primarily because it is effectively
computable.

\begin{thm}
  Let $E : y^2 = x^3 + ax^2 + bx$ and write $b = p_1^{e_1} \cdots p_t^{e_t}$.
  Then $Sel \simeq \{b_1 = \pm p_1^{a_1} \cdots p_t^{a_t} \, | \, a_i = 0 \text { or }
  1 \text{ and the equation } N^2 = b_1M^4 + \frac{b}{b_1}e^4 \text{ has a solution.}\}$

\begin{defn}[Tate-Shafarevich Group]
  The Tate-Shafarevich group of the elliptic curve $E$ is defined
  $$ \TS(E / \QQ) = \text{Ker} \big((H^1(\QQ, \bar{E}) \ra \prod\limits_\nu H^1 (\QQ_\nu,
  \bar{E})[\psi]\big) $$
\end{defn}


\end{document}
