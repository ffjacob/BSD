\message{ !name(galois.tex)}% Latex template
\documentclass[11pt,a4paper]{amsart}
\usepackage{amsmath,amsfonts,amsthm,amssymb,color,dsfont}

\usepackage{hyperref}
\usepackage[alphabetic,initials]{amsrefs}

%%%%%%%%%%%%%%%%%%%%%%%%% preamble

% theorem-type environments
\theoremstyle{plain}
\newtheorem{prop}{Proposition}[section]
\newtheorem{thm}[prop]{Theorem}
\newtheorem{cor}[prop]{Corollary}
\newtheorem{lemma}[prop]{Lemma}

\theoremstyle{definition}
\newtheorem{example}[prop]{Example}
\newtheorem{defn}[prop]{Definition}

\theoremstyle{remark}
\newtheorem{remark}[prop]{Remark}

\numberwithin{equation}{section}

% user-defomed macros
\newcommand{\defin}{\textbf}
\newcommand{\CC}{{\mathbb C}}
\newcommand{\cov}{{\operatorname{cov}}}
\newcommand{\eE}{{\mathcal E}}
\newcommand{\NN}{{\mathbb N}}
\newcommand{\PP}{{\mathbb P}}
\newcommand{\ZZ}{{\mathbb Z}}
\renewcommand{\SS}{{\mathbb S}}
\newcommand{\DD}{{\mathbb D}}
\newcommand{\RR}{{\mathbb R}}
\newcommand{\QQ}{{\mathbb Q}}
\newcommand{\rR}{{\mathcal R}}
\newcommand{\OO}{{\mathcal O}}
\newcommand{\p}{\partial}
\newcommand{\mM}{{\mathcal M}}
\newcommand{\pP}{{\mathcal P}}
\newcommand{\iI}{{\mathcal I}}
\newcommand{\jJ}{{\mathcal J}}
\newcommand{\uU}{{\mathcal U}}
\newcommand{\sS}{{\mathfrak S}}
\newcommand{\1}{{\mathds 1}}
\newcommand{\Crit}{\operatorname{Crit}}
% colours
\definecolor{red}{rgb}{1,0,0}
\newcommand{\red}[1]{{\color{red}#1}}
\newcommand{\GKK}{{G_{\bar{K} : K}}}
\newcommand{\st}{{\text{s.t.}}}
\newcommand{\ra}{\rightarrow}

\title[Galois Cohomology]{Galois Cohomology}

\begin{document}

\message{ !name(galois.tex) !offset(86) }
\section{Applications to Elliptic Curves}
A crucial step in the proof of the Mordell-Weil theorem, enough to show that
the rank of an elliptic curve over $\QQ$ is finite, is the study of the size
of the quotient $E(\QQ) / 2 E(\QQ)$. In curves with at least one 2-torsion
point, this can be done with the help of an isogeny.

Let $$E : y^2 = x^3 + ax^2 + bx$$ be an elliptic curve over $\QQ$. We know $E$
has the rational points $\OO, T = (0,0)$.

We also consider the curve
\[ \bar{E} : y^2 = x^3 + \bar{a}x^2 + \bar{b}x \]
where $\bar{a} = -2a$ and $\bar{b} = a^2 - 4b$. Repeating this process yields
the curve
\[ \bar{\bar{E}} : y^2 = x^3 +  4ax^2 + 16bx\]
which is birationally equivalent to $E$ by the transformation $y \mapsto 8y,
x \mapsto 4x$.

\begin{prop}
  Let $E, \bar{E}$ be as above. The maps $\phi : E \rightarrow \bar{E}$ and
  $\psi : \bar{E} \rightarrow E$ defined by
  \[\phi(P) =
    \begin{cases}
      (\frac{y^2}{x^2}, \frac{y(x^2-b)}{x^2}), & \text{ if } P \neq \OO, T \\
      \bar{\OO}, & \text{ if } P = \OO \text{ or } P = T
    \end{cases}\]
  and
  \[\psi(P) =
    \begin{cases}
      (\frac{\bar{y}^2}{4\bar{x}^2}, \frac{\bar{y}(\bar{x}^2-\bar{b})}{8\bar{x}^2}),
      & \text{ if } P \neq \bar{\OO}, \bar{T} \\
      \bar{\OO}, & \text{ if } \bar{P} = \bar{\OO} \text{ or } \bar{P} = \bar{T}
    \end{cases}\]
  are elliptic curve homomorphisms, $\text{Ker}(\phi) = \{\OO, T\}$ and
  \[ \psi \circ \phi (P) = 2P , \text{ for all points } P \in E.\]
\end{prop}
Proof: Silverman and Tate

\begin{lemma}
  If $E / \psi(\bar{E}))$ and $\bar{E} / \phi(E)$ are finite, then so is $E/2E$
\end{lemma}
Proof: Silverman and Tate. \qedhere

We are thus led to consider the quotient $E(\QQ) / \psi(\bar{E}(\QQ))$ (the
other one can be treated identically).

By the proposition, we have a short exact sequence of $G_{\QQ}$-modules
\[ 0 \rightarrow \{\OO, T\} \rightarrow \bar{E}(\QQ) \rightarrow E(\QQ)
  \rightarrow 0\]
and $\{\OO, T\} \simeq \ZZ/2\ZZ$. Taking Galois cohomology we get the long
exact sequence

\[ 0 \rightarrow \ZZ / 2\ZZ \rightarrow \bar{E}(\QQ) \rightarrow E(\QQ)
  \rightarrow H^1(\QQ, \ZZ/2\ZZ) \rightarrow H^1(\QQ, \bar{E}) \rightarrow
  H^1(\QQ, E)\]
This in turn gives us the short exact sequence

\[ 0 \rightarrow \frac{E(\QQ)}{\psi(\bar{E} (\QQ))} \rightarrow
  H^1(\QQ, \ZZ / 2\ZZ) \rightarrow H^1(\QQ, \bar{E}(\QQ))[\psi] \rightarrow 0\]

To get the order of $E(\QQ) / \psi(\bar{E}(\QQ))$ we need to know the order of
$H^1(\QQ, \ZZ/2\ZZ)$.

\begin{prop}
  $H^1(\QQ, \ZZ/2\ZZ) \simeq \QQ^{\times} / (\QQ^{\times})^2$
\end{prop}
Proof: Consider the exact sequence of Galois modules
\[0 \rightarrow \mu_2 \rightarrow \bar{\QQ}^{\times} \xrightarrow[]{2} \bar{\QQ}^{\times} \ra
  \, 0 \]
where $\mu_2 = \text{Gal} (\QQ(i) : \QQ) \simeq \ZZ / 2\ZZ.$

Taking cohomology gives the long exact sequence
\[ 0 \ra \mu_2 \ra \bar{\QQ}^{\times} \xrightarrow[]{2} \bar{\QQ}^{\times} \ra H^1(\QQ, \ZZ/2\ZZ)
  \ra H^1(\QQ, \bar{\QQ}^{\times})\]
and $H^1(\QQ, \bar{\QQ}^{\times}) \simeq 0$ by Hilbert's Theorem 90. Thus,
\[H^1(\QQ, \ZZ/2\ZZ) \simeq \QQ^{\times}/(\QQ^{\times})^2. \]



\message{ !name(galois.tex) !offset(84) }

\end{document}
