\message{ !name(calculations.tex)}% % Latex template
 \documentclass[11pt,a4paper]{amsart}
% \usepackage{amsmath,amsfonts,amsthm,amssymb,color,dsfont}

\usepackage{hyperref}
\usepackage[alphabetic,initials]{amsrefs}

%%%%%%%%%%%%%%%%%%%%%%%%% preamble

% theorem-type environments
\theoremstyle{plain}
\newtheorem{prop}{Proposition}[section]
\newtheorem{thm}[prop]{Theorem}
\newtheorem{cor}[prop]{Corollary}
\newtheorem{lemma}[prop]{Lemma}

\theoremstyle{definition}
\newtheorem{example}[prop]{Example}

\newtheorem{defn}[prop]{Definition}

\theoremstyle{remark}
\newtheorem{remark}[prop]{Remark}

\numberwithin{equation}{section}

% user-defomed macros
\newcommand{\defin}{\textbf}
\newcommand{\CC}{{\mathbb C}}
\newcommand{\cov}{{\operatorname{cov}}}
\newcommand{\eE}{{\mathcal E}}
\newcommand{\NN}{{\mathbb N}}
\newcommand{\PP}{{\mathbb P}}
\newcommand{\ZZ}{{\mathbb Z}}
\renewcommand{\SS}{{\mathbb S}}
\newcommand{\DD}{{\mathbb D}}
\newcommand{\RR}{{\mathbb R}}
\newcommand{\QQ}{{\mathbb Q}}
\newcommand{\OO}{{\mathcal O}}
\newcommand{\FF}{{\mathbb F}}
\newcommand{\EFp}{{\tilde{E}(\FF_p)}}
\newcommand{\EFt}{{\tilde{E}(\FF_2)}}
\newcommand{\EQp}{{E(\QQ_p)}}
\newcommand{\rR}{{\mathcal R}}
\newcommand{\p}{\partial}
\newcommand{\mM}{{\mathcal M}}
\newcommand{\pP}{{\mathcal P}}
\newcommand{\iI}{{\mathcal I}}
\newcommand{\jJ}{{\mathcal J}}
\newcommand{\uU}{{\mathcal U}}
\newcommand{\sS}{{\mathfrak S}}
\newcommand{\1}{{\mathds 1}}
\newcommand{\Crit}{\operatorname{Crit}}

\title[Latex Template]{Calculations}

\begin{document}

\message{ !name(calculations.tex) !offset(22) }
\section{Coefficients of Theta Series}
Tunnell's theorem requires us to consider the quantities 
\[  A_n = \#\{(x,y,z) | n = 2x^2 + y^2 + 32z^2\} \]
\[  B_n = \#\{(x,y,z) | n = 2x^2 + y^2 + 8z^2\} \]
\[  C_n = \#\{(x,y,z) | n = 8x^2 + 2y^2 + 64z^2\} \]
\[  D_n = \#\{(x,y,z) | n = 8x^2 + 2y^2 + 16z^2\} \]

depending on whether $n$ is odd or even. For our calculations we'll
need to compute $A_n - \frac{1}{2}B_n$ and $C_n - \frac{1}{2}D_n$ modulo 4.

\begin{thm}
  $2A_p - B_p \equiv \begin{cases} 0, & p \equiv 1, 5, 7 \, (8) \\
    2, & p \equiv 3 \, (8) \end{cases}
  \, (4)$
\end{thm}
Proof: In computing $A_P$ and $B_P$ we only need to consider solutions
where at least 1 of $x, y, z$ is 0, since if neither is 0, all of $\pm x, \pm y, \pm z$
are solutions, and hence they together don't make a contribution to the total
number of solutions modulo 8. Furthermore, since we only need to consider $A_n$
modulo 4, we can also ignore solutions where exactly 1 of $x, y z$ are 0. But
there are no remaining solutions since p is prime and if 2 of $x, y, z$ are 0 we
arrive at the contradiction $x^2 = p$, so we have $A_p \equiv 0 \, (4)$.

With this we're left to consider $\frac{1}{2}B_p \, (4)$, or $B_p \, (8)$. Again we can
ignore solutions where all of $x, y, z$ are 0, so we have, by inclusion-exclusion

\( A_p \equiv \#\{ 2x^2 + y^2 = p\} + \#\{y^2 + 8z^2 = p\} + \#\{2x^2 + 8z^2 = p\} \\
\equiv \#\{ 2x^2 + y^2 = p\} + \#\{y^2 + 8z^2 = p\} \, (8)\)
since p is odd. To find the remaining quantities we need some algebraic number
theory.

\begin{lemma}
  For $p$ odd, $\#\{2x^2 + y^2 = p\} = \begin{cases} 4,
    p \equiv 1 \, (8) \\
    1, p \equiv 3 \, (8) \\
    0, else \end{cases}$
\end{lemma}
Proof: We must have $2x^2 + y^2 \equiv 0 \, (p)$, so we only have solutions if
$(\frac{-2}{p}) = 1$, or equivalently if $p \equiv 1, 3 \, (8)$. Since
$\ZZ[\sqrt{2}]$ is a PID, if $p \equiv 1 \, (8), p = \pi \bar{\pi}$ for some
prime $\pi = x + y \sqrt{2}$ where $x^2 + 2y^2 = p$, so $x$ is odd and $x^2
\equiv 1 \, (8)$
If $y$ is also odd, $1 + 2y^2 \equiv 1 \, (8) \implies y^2 \equiv 1 \, (8)$, and
we have $x^2 + 2y^2 \equiv 3 \, (8)$, a contradiction.
If $y = 2k$ is even, we have $x^2 + 2y^2 = x^2 + 8y^3$ 

To conclude the proof of the theorem, we can deduce from the lemma that
\begin{itemize}
  \item If $p \equiv 1 \, (8), A_p \equiv 4+4 \equiv 0 \, (8)$.
  \item If $p \equiv 3 \, (8), A_p \equiv 4 + 0 \equiv 4 \, (8)$.
  \item If $p \equiv 5, 7 \, (8), A_p \equiv 0 + 0 \equiv 0 \, (8)$.
\end{itemize} 
Looking at $A_p - \frac{1}{2} B_p$ modulo 4 gives the result.

\message{ !name(calculations.tex) !offset(178) }

\end{document} 