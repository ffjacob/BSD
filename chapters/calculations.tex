% % Latex template
 \documentclass[11pt,a4paper]{amsart}
% \usepackage{amsmath,amsfonts,amsthm,amssymb,color,dsfont}

\usepackage{hyperref}
\usepackage[alphabetic,initials]{amsrefs}

%%%%%%%%%%%%%%%%%%%%%%%%% preamble

% theorem-type environments
\theoremstyle{plain}
\newtheorem{prop}{Proposition}[section]
\newtheorem{thm}[prop]{Theorem}
\newtheorem{cor}[prop]{Corollary}
\newtheorem{lemma}[prop]{Lemma}

\theoremstyle{definition}
\newtheorem{example}[prop]{Example}

\newtheorem{defn}[prop]{Definition}

\theoremstyle{remark}
\newtheorem{remark}[prop]{Remark}

\numberwithin{equation}{section}

% user-defomed macros
\newcommand{\defin}{\textbf}
\newcommand{\CC}{{\mathbb C}}
\newcommand{\cov}{{\operatorname{cov}}}
\newcommand{\eE}{{\mathcal E}}
\newcommand{\NN}{{\mathbb N}}
\newcommand{\PP}{{\mathbb P}}
\newcommand{\ZZ}{{\mathbb Z}}
\renewcommand{\SS}{{\mathbb S}}
\newcommand{\DD}{{\mathbb D}}
\newcommand{\RR}{{\mathbb R}}
\newcommand{\QQ}{{\mathbb Q}}
\newcommand{\OO}{{\mathcal O}}
\newcommand{\FF}{{\mathbb F}}
\newcommand{\EFp}{{\tilde{E}(\FF_p)}}
\newcommand{\EFt}{{\tilde{E}(\FF_2)}}
\newcommand{\EQp}{{E(\QQ_p)}}
\newcommand{\rR}{{\mathcal R}}
\newcommand{\p}{\partial}
\newcommand{\mM}{{\mathcal M}}
\newcommand{\pP}{{\mathcal P}}
\newcommand{\iI}{{\mathcal I}}
\newcommand{\jJ}{{\mathcal J}}
\newcommand{\uU}{{\mathcal U}}
\newcommand{\sS}{{\mathfrak S}}
\newcommand{\1}{{\mathds 1}}
\newcommand{\Crit}{\operatorname{Crit}}

\title[Latex Template]{Calculations}

\begin{document}

\maketitle

\section{Introduction}

The Birch Swinnerton-Dyer conjecture states that the Hasse-Weil L-function
$L(E,s)$ of the elliptic curve $E$ can be written as
\[L(E,s) = C (s-1)^r + O((s-1)^2)\] where
\[C = \frac{|\text{Sha}(E)| \, \Omega_E R_E}{|E(\QQ)_{\tors}|^2} \prod_p c_p\]
In particular, near 1, it predicts the value of
\[0^r |\text{Sha}(E)| \prod_p c_p \in \ZZ\].

\begin{defn}
  From now on we let $E_n : y^2 = x^3 - n^2x$
  be the family of quadratic twists of the elliptic
  curve $E_1 : y^2 = x^3 - x$.
\end{defn}

From Tunnell's theorem we know that $L(E_n,s)$ has a particularly simple
form in terms of Theta functions. In this section we'll show that they
can be used to predict whether $0^r |\text{Sha}(E)$ is even or odd, and that
this agrees with what's the BSD conjecture.


\section{Coefficients of Theta Series}
Tunnell's theorem requires us to consider the quantities 
\[  A_n = \#\{(x,y,z) | n = 2x^2 + y^2 + 32z^2\} \]
\[  B_n = \#\{(x,y,z) | n = 2x^2 + y^2 + 8z^2\} \]
\[  C_n = \#\{(x,y,z) | n = 8x^2 + 2y^2 + 64z^2\} \]
\[  D_n = \#\{(x,y,z) | n = 8x^2 + 2y^2 + 16z^2\} \]

depending on whether $n$ is odd or even. For our calculations we'll
need to compute $A_n - \frac{1}{2}B_n$ and $C_n - \frac{1}{2}D_n$ modulo 4.

\begin{thm}
  $2A_p - B_p \equiv \begin{cases} 0, & p \equiv 1, 5, 7 \, (8) \\
    2, & p \equiv 3 \, (8) \end{cases}
  \, (4)$
\end{thm}
Proof: In computing $A_P$ and $B_P$ we only need to consider solutions
where at least 1 of $x, y, z$ is 0, since if neither is 0, all of $\pm x, \pm y, \pm z$
are solutions, and hence they together don't make a contribution to the total
number of solutions modulo 8. Furthermore, since we only need to consider $A_n$
modulo 4, we can also ignore solutions where exactly 1 of $x, y z$ are 0. But
there are no remaining solutions since p is prime and if 2 of $x, y, z$ are 0 we
arrive at the contradiction $x^2 = p$, so we have $A_p \equiv 0 \, (4)$.

With this we're left to consider $\frac{1}{2}B_p \, (4)$, or $B_p \, (8)$. Again we can
ignore solutions where all of $x, y, z$ are 0, so we have, by inclusion-exclusion

\( A_p \equiv \#\{ 2x^2 + y^2 = p\} + \#\{y^2 + 8z^2 = p\} + \#\{2x^2 + 8z^2 = p\} \\
\equiv \#\{ 2x^2 + y^2 = p\} + \#\{y^2 + 8z^2 = p\} \, (8)\)
since p is odd. To find the remaining quantities we need some algebraic number
theory.

\begin{lemma}
  For $p$ odd, $\#\{2x^2 + y^2 = p\} = \begin{cases} 4,
    p \equiv 1 \, (8) \\
    1, p \equiv 3 \, (8) \\
    0, else \end{cases}$
\end{lemma}
Proof: We must have $2x^2 + y^2 \equiv 0 \, (p)$, so we only have solutions if
$(\frac{-2}{p}) = 1$, or equivalently if $p \equiv 1, 3 \, (8)$. Since
$\ZZ[\sqrt{2}]$ is a PID, if $p \equiv 1 \, (8), p = \pi \bar{\pi}$ for some
prime $\pi = x + y \sqrt{2}$ where $x^2 + 2y^2 = p$, so $x$ is odd and $x^2
\equiv 1 \, (8)$
If $y$ is also odd, $1 + 2y^2 \equiv 1 \, (8) \implies y^2 \equiv 1 \, (8)$, and
we have $x^2 + 2y^2 \equiv 3 \, (8)$, a contradiction.
If $y = 2k$ is even, we have $x^2 + 2y^2 = x^2 + 8y^3$ 

To conclude the proof of the theorem, we can deduce from the lemma that
\begin{itemize}
  \item If $p \equiv 1 \, (8), A_p \equiv 4+4 \equiv 0 \, (8)$.
  \item If $p \equiv 3 \, (8), A_p \equiv 4 + 0 \equiv 4 \, (8)$.
  \item If $p \equiv 5, 7 \, (8), A_p \equiv 0 + 0 \equiv 0 \, (8)$.
\end{itemize} 
Looking at $A_p - \frac{1}{2} B_p$ modulo 4 gives the result.

\section{Tamagawa Numbers}

The Hasse-Weil L-function of an elliptic curve doesn't capture any information
about the primes where the curve is singular. The arithmetic data at those
primes enters the Birch Swinnerton-Dyer conjecture by means of the Tamagawa
Numbers. Let
$$ E : y^2 = x^3 - n^2 x $$.
If $p \not| \Delta E = 4n^2$ then the reduction $\tilde{E} : y^2 \equiv x^3 - n^2 x \,
(p)$ is
also an elliptic curve. We have

\begin{thm}[Reduction Modulo p]
  The reduction map $E(\QQ) \rightarrow \tilde{E}(\FF_p)$ is a group homomorphism.
\end{thm}

If $p | \Delta E$ then $\tilde{E}(\FF_p)$ is not a group, but
$\tilde{E}(\FF_p)_{ns} := \{P \in \tilde{E}(\FF_p) | P \text{is nonsingular}\}$
is still a group.

Looking at $E$ in the completion $\QQ_p$ we still have a map $\EQp \rightarrow
\EFp$.
If we let $\EQp_0$ be the preimage of $\EFp_{ns}$ we get a group homomorphism
$$ \EQp_0 \rightarrow \EFp_{ns} $$.

\begin{defn}
  The Tamagawa Number of $E$ at $p$ is the number
  $$c_p = \big| \frac{\EQp}{\EQp_0}
  \big|.$$
  If $p \not| \Delta$ then $c_p = 1$.
\end{defn}  

\begin{thm}
  The Tamagawa Number of $E_p$ at $p$ is 4.
\end{thm}
Proof: Working in $\QQ_p$, if $x \equiv 0$ then $y^2 = x^3 - p^2 x \equiv 0 \,
(p^2) \implies y \equiv 0 \, (p)$, so $(x, y) \equiv (0,0) \, (p)$.
If $x$ has nonpositive valuation, $x = p^{-n} u, u \in \ZZ_p^{\times}$ and $y^2 =
p^{-3n}u^3 - p^{2-n} u$, so $n$ must be even. 

Let $n = 2m$. Then $p^{6m} y^2 = u^3 - p^{2+2m}u \implies y = p^{-3m}v, v \in
\ZZ_p^{\times}$. Since $u, v \not\equiv 0 
\, (p)$, this has solutions if and only if $(\frac{u}{p}) = 1$, so $x$ is the
reciprocal of a square in $\ZZ_p$.

Setting $x = \frac {1}{r^2}, r^2 \in \ZZ_p$ we get $y^2 = \frac{1}{r^6} -
\frac{p^2}{r^2} = \frac{1 - p^2r^4}{r^6}$, so we have a bijection
\[\ZZ_p \rightarrow \EQp_0\]
\[r \mapsto (\frac{1}{r^2}, \frac{\sqrt{1-p^2r^4}}{r^3})\]
\[0 \mapsto \mathcal O\]
To prove the theorem we will need the following lemma:

\begin{lemma}
  The torsion points $\OO, (0,0), (p,0), (-p,0)$ are a complete set of
  representatives for $\frac{\EQp}{\EQp_0}$.
\end{lemma}
Proof: Let $P(r) = (\frac{1}{r^2}, \frac{\sqrt{1-p^2r^4}}{r^3}) = (x,y)$. We will
compute $S := P(r) + Q$ for each torsion point $Q$.
\begin{itemize}
  \item $Q = \OO$: $P(r) + \OO = P(r)$.
  \item $Q = (0,0)$:
    We have $\lambda = \frac{y}{x}$, so
    \begin{equation*}
      \begin{split}
        x(S)
        &= \lambda^2 - x - 0 =
        \frac{\frac{1 - p^2r^4}{r^6}}{\frac{1}{r^4}} - \frac{1}{r^2} \\
        &= -p^2r^2
      \end{split}
    \end{equation*}
  \item $Q = (p, 0)$: We have $\lambda = \frac{y}{x-p}$, so
    \begin{equation*}
      \begin{split}
        x(S)
        &= \lambda^2 - x - p = 
        \frac{\frac{1 - p^2r^4}{r^6}}{\frac{(1-pr^2)^2}{r^4}} -
        \frac{1}{r^2} - p \\
        &= \frac{1-p^2r^4}{r^2(1-pr^2)^2} - \frac{1+pr^2}{r^2} \\
        &= -p + \frac{2p}{1-pr^2}
      \end{split}
    \end{equation*}
  \item $Q = (-p, 0)$: We have $\lambda = \frac{y}{x+p}$, so
    \begin{equation*}
      \begin{split}
        x(S)
        &= \lambda^2 - x + p = 
        \frac{\frac{1 - p^2r^4}{r^6}}{\frac{(1+pr^2)^2}{r^4}} -
        \frac{1}{r^2} + p \\
        &= p - \frac{2p}{1+pr^2}
      \end{split}
    \end{equation*}
\end{itemize}

Now let $S = (x,y)$.
If $x \equiv 0 \, (p^2)$, $x = p^2 t, t \in \ZZ_p$ we want to show that $x = p^2r^2, r \in \ZZ_p$, so that $S$ lies
in the coset $P(r) + (0,0)$. Note that
\begin{equation*}
  \begin{split}
    y^2 &= x^3 - p^2x \\
    &= x(x+p)(x-p) \\
    x &= \frac{y^2}{(x+p)(x-p)} \\
    &= \frac{y^2}{p^2(1+tp)(1-tp)}
  \end{split}
\end{equation*}
and $1 \pm tp$ are squares in $\QQ_p$, so indeed $x$ is a square and we can set
$r = \sqrt{\frac{x}{p^2}}$.

If $x \equiv p \, (p^2), x = p + p^2t$ we want to solve
\[p - \frac{2p}{1+pr^2} = x = p + p^2t \] or
\[ \frac{-2}{1+pr^2} = pt\]
to show that $S$ lies in the coset $P(r)+(-p,0)$.
\begin{equation*}
  \begin{split}
    1+pr^2 &= \frac{-2}{pt} \\
    r^2 &= \frac{-2-pt}{p^2t}
  \end{split}
\end{equation*}.
Now
\[ x = \frac{y^2}{(x+p)(x-p)} = \frac{y^2}{(p^2t+2p)(p^2t)}
= \frac{y^2}{p^2t(2+pt)p}\]
\[ \implies x = \frac{y^2}{-r^2 p}\]
\[ \implies r^2 = \frac{y^2}{-px} = \frac{y^2}{p^2(-1-pt)}\]


If $x \equiv -p \, (p^2), x = -p + p^2t$ and we solve
\[ -p + \frac{2p}{1-pr^2} = x = -p + p^2t\] or
\[ \frac{2}{1-pr^2} = pt\]
\[ \implies r^2 = \frac{pt - 2}{p^2t}\]

Now
\[ x = \frac{y^2}{(x+p)(x-p)} = \frac{y^2}{(p^2t-2p)(p^2t)}
= \frac{y^2}{p^2t(pt-2)p}\]
\[\implies x = \frac{y^2}{r^2p}\]
\[ r^2 = \frac{y^2}{xp} = \frac{y^2}{p^2(1+pt)}\]
which is a square, so again we can solve for $r$.

\begin{thm}
  The Tamagawa Number of $E_P$ at 2 is 2.
\end{thm}
Proof: By checking all possibilities, we know that
\[\EFt = \{\OO, (0,0), (1,0)\},\]
with $(1,0)$ being the singular point. Let $\pi : E(\QQ_2)_0 \rightarrow
\EFt_{ns}$ be the projection map. 
or any $P = (x,y) \in E(\QQ_2)$, if $\pi(P) = (1,0)$ then $\pi(P * (p,0)) \neq (p,0)$.
If $x$ is a unit, let $(x,y) * (p,0) = (x',y')$. We want to show that $x'$ is
not a unit. Suppose it is, and let $\lambda$ be the gradient of the line through
$(x,y), (p,0), (x',y')$, so that $\lambda = \frac{y}{x-p}$.
Then $x + p + x' = \lambda^2$, so $\lambda \not\equiv 0 \, (2)$, so $\lambda$ is
a unit. We also have
\[
  \begin{split}
     y^2 &= x^3 - p^2x \\
     \implies \lambda^2(x-p)^2 &= x (x+p) (x-p) \\
     \implies u^2 &= \frac{x (x+p)}{x-p} 
   \end{split}
\]
implying $\frac{x+p}{x-p}$ is a unit, which is a contradiction, since
$\nu_2(\frac{x+p}{x-p}) = $

\end{document} 