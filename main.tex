\documentclass[12pt, a4paper]{amsart}
\usepackage{amsmath,amsfonts,amsthm,amssymb,color,dsfont,tikz-cd,minitoc,pifont,tabu,biblatex}
\addbibresource{references.bib}

\usepackage{hyperref}

\begin{document}

% Latex template
%%%%%%%%%%%%%%%%%%%%%%%%% preamble

% theorem-type environments
\theoremstyle{plain}
\newtheorem{prop}{Proposition}[subsection]
\newtheorem{thm}[prop]{Theorem}
\newtheorem{cor}[prop]{Corollary}
\newtheorem{lemma}[prop]{Lemma}
\newtheorem{conj}[prop]{Conjecture}

\theoremstyle{definition}
\newtheorem{example}[prop]{Example}
\newtheorem{defn}[prop]{Definition}

\theoremstyle{remark}
\newtheorem{remark}[prop]{Remark}

\numberwithin{equation}{subsection}

% user-defomed macros
\newcommand{\defin}{\textbf}
\newcommand{\CC}{{\mathbb C}}
\newcommand{\cov}{{\operatorname{cov}}}
\newcommand{\eE}{{\mathcal E}}
\newcommand{\NN}{{\mathbb N}}
\newcommand{\PP}{{\mathbb P}}
\newcommand{\ZZ}{{\mathbb Z}}
\renewcommand{\SS}{{\mathbb S}}
\newcommand{\DD}{{\mathbb D}}
\newcommand{\RR}{{\mathbb R}}
\newcommand{\QQ}{{\mathbb Q}}
\newcommand{\rR}{{\mathcal R}}
\newcommand{\OO}{{\mathcal O}}
\newcommand{\p}{\partial}
\newcommand{\mM}{{\mathcal M}}
\newcommand{\pP}{{\mathcal P}}
\newcommand{\iI}{{\mathcal I}}
\newcommand{\jJ}{{\mathcal J}}
\newcommand{\uU}{{\mathcal U}}
\newcommand{\sS}{{\mathfrak S}}
\newcommand{\1}{{\mathds 1}}
\newcommand{\Crit}{\operatorname{Crit}}
% colours
\definecolor{red}{rgb}{1,0,0}
\newcommand{\red}[1]{{\color{red}#1}}
\newcommand{\GKK}{{G_{\bar{K} : K}}}
\newcommand{\st}{{\text{s.t.}}}
\newcommand{\ra}{\rightarrow}
\newcommand{\Sel}{\text{Sel}}
\newcommand{\Sha}{\text{Sha}}
\newcommand{\TS}{\text{TS}}
\newcommand{\Eb}{\bar{E}}
\newcommand{\EQ}{E(\QQ)}
\newcommand{\cmark}{\ding{51}}
\newcommand{\xmark}{\ding{55}}
\newcommand{\EFp}{{\tilde{E}(\FF_p)}}
\newcommand{\EFt}{{\tilde{E}(\FF_2)}}
\newcommand{\EQp}{{E(\QQ_p)}}
\newcommand{\FF}{\mathbb{F}}


\title{
  {The Birch Swinnerton-Dyer Conjecture}\\
}
\author{Felipe Jacob}
\date{5 December 2016}

\tableofcontents
\pagebreak

\maketitle

\section{Elliptic Curve Invariants and the BSD Conjecture}

One of the major topics in modern number theory is the study of rational
solutions to cubic equations in two variables, called elliptic curves.
In this project we will
investigate the relationships between the arithmetic invariants of such
equations and certain analytic functions. Such relationships give rise to
many conjectures with powerful implications, a few of which we will discuss.
We will first define the objects of interest and their invariants, and then
continue with the main calculations.

\subsection{Basic definitions and the group law}

\begin{defn}
  If $K$ is a perfect field,
  we say an \defin{elliptic curve} over $K$ is a smotth cubic projective
  curve $E$ defined over $K$ with at least one rational point $\OO \in E(K)$
  that we call the \defin{point at infinity}.
\end{defn}
With the help of algebraic geometry, we know that such curves can be put
into particularly simple forms.

\begin{prop}
  If $E$ is an elliptic curve over $K$, then it is birationally equivalent
  to the curve
  \begin{equation} \label{eq:weier2}
    E: y^2 + a_1xy + a_3y = x^3 + a_2x^2 + a_4x + a_6
  \end{equation}
  with $a_1, a_2, a_3, a_4 \in K.$
  Furthermore, when $\text{char}(K) \neq 2, E$ (such as in number fields)
  can be written in the particularly simple form
  \begin{equation} \label{eq:weier}
    E: y^2 = x^3 + ax + b
  \end{equation}
  with $a, b \in K$.
\end{prop}

\begin{proof}
  \cite[See][Chapter III, pages 42-43]{arithmetic}.
\end{proof}

From now on, unless said otherwise, we will take $K = \QQ$ and $E$ given
by \autoref{eq:weier}.

In this project we will be particularly interested in the following family
of elliptic curves

\begin{defn}
  The family of \defin{quadratic twists of $E : y^2 = x^3 - x$} is the family of curves
  \begin{equation} \label{eq:twist}
    E_n : y^2 = x^3 - n^2 x
  \end{equation}
  where $n$ is a positive integer.
\end{defn}

Elliptic curves are of particular interest primarily because, besides being
a variety, they also form an abelian group.

\begin{prop}
  If $P_1 = (x_1,y_1), P_2 = (x_2,y_2) \in E, P_1, P_2 \neq \OO$, let
  $$\lambda = \frac{y_1-y_2}{x_1-x_2}.$$
  We then set
  \[P_1 + P_2 = (x_3,y_3)\]
  where
  \begin{equation} \label{eq:addx}
    x_3 = \lambda^2 - a - x_1 - x_2
  \end{equation}
  \begin{equation} \label{eq:addy}
    y_3 = \lambda x_2 + (y_1 - \lambda x_1)
  \end{equation}
  Under this addition law, $E(\QQ)$ is an abelian group.
\end{prop}

\begin{proof}
  The easiest proof uses an equivalence between elliptic curves and lattices
  in $\CC$. \cite[See][Chapter I, sections 6-7]{modular}.
\end{proof}

\begin{defn}
  By the classification theorem of abelian groups, we can write
\[\EQ =
  \text{Tors}(E) \times \ZZ^r. \]
Here $\text{Tors}(E)$ is the set of points in
$\EQ$ with finite
order, called the \defin{torsion group} of $E$,
and $r$ is a not necessarily finite number called the \defin{rank of $E$}.
We will also ocasionally write $r = R_E$.
\end{defn}

The following results about $\EQ$ were discussed in the course MATH3705,
and their proofs can be found in \cite{rational}, chapters II and III.

\begin{thm}[Nagell-Lutz Theorem] \label{thm:nagelllutz}
  Let
  \[E: y^2 = f(x) = x^3 + ax + b\]
  be an elliptic curve over $\QQ$, and let $\Delta = -4a^3 -27b^2$ be the
  polynomial discriminant of $f(x)$. If $P = (x,y)$ is a point of finite order,
  then either $y = 0$, in which case $P$ has order 2, or we must have $y^2 | \Delta$.
\end{thm}

\begin{proof}
  \cite[See][Chapter II, pages 49-56]{rational}.
\end{proof}

\begin{thm}[Mordell-Weil Theorem]
  $r$ is a finite natural number.
\end{thm}  

\begin{proof}
  \cite[See][Chapter III, pages 63-88]{rational}. Also see the discussion on
  \autoref{sec:applications}.
\end{proof}


\subsection{The Hasse-Weil L-function}

As with other projective varieties, we can define certain generating functions
that capture the behaviour of $E$ locally, i.e. when $K = \FF_p$.

\begin{defn}
  The \defin{local zeta function of $E$ at $p$} is the formal power series
  \[Z(E, p, z) = \exp \big( \sum\limits_{r=1}^{\infty} \# E(\FF_{p^r}) \frac{z^r}{r} \big)\]
  where $\# E(\FF_{p^r})$ is the number of points of $E$ in the (unique) field
  of $p^r$ elements.
\end{defn}

For our family $E_n$ the local zeta function is explicitly given by the
following proposition

\begin{prop}
  If $p \not| \,\, 2n$, then
  \[Z(E_n, p, z) = \frac{1 - 2az + pz^2}{(1-z)(1-pz)} \]
  where $a = \Re(\alpha)$, and $\alpha$ is the complex number
  \[ \alpha =
    \begin{cases}
      i \sqrt{p} & \text{ if } p \equiv 3 \, (4) \\
      p & \text{ if } p \equiv 1 \, (4)
    \end{cases}.\]
\end{prop}

\begin{proof}
  \cite[See][Chater II-2, pages 59-61]{modular}.
\end{proof}

Encouraged by the local-global principles that work on quadratic curves, we
amalgamate local zeta functions into a single object, hoping it will give us
information about $E(\QQ)$.

\begin{defn}
  The \defin{Hasse-Weil L-function} of $E$ is written
  \begin{equation} \label{eq:lfunction}
    L(E, s) = \frac{\zeta(s) \zeta(s-1)}{\prod_p Z(E, p, p^{-s})}
  \end{equation}
  where $\zeta(s)$ is the Riemann-zeta function. The numerator is introduced
  to clear the denominator of the local zeta functions.
\end{defn}

\subsection{The Regulator and Sha}

Before we are ready to state the main conjecture, we need to define a few
other invariants of $E$. 

Firstly, we will define an invariant that controls the growth of the complexity
of the points of $E$ as we take higher and higher multiples.

\begin{defn}[Regulator] 
  The \defin{regulator $\Omega_{E_n}$ of $E_n$} is the real number
  \[ \Omega_{E_n} = \frac{1}{\sqrt{n}} \int_{E(\RR)} \frac{dx}{2y} =
    \frac{1}{2 \sqrt{n}} \int_1^\infty \frac{dx}{\sqrt{x^3-x}}.\]
\end{defn}
\begin{remark}
  The complete definition of the regulator in terms of the Neron-Tate pairing
  won't be given here. It was discussed in MATH3703.
  See also
  \cite[Section 13-7, page 126]{Granville} and
  \cite[Section VIII.9, pages 252-253]{arithmetic}.
\end{remark}

The next crucial invariant measures in some sense the hardness of working
out the rank of $E$ by using local methods.

\begin{defn}[Tate-Shafarevich Group]
  We need more background material before we're ready to define $\Sha(E)$.
  See Definition \autoref{defn:sha}.
\end{defn}

\subsection{Tamagawa Numbers}

Finally, the product in \autoref{eq:lfunction} doesn't really capture any information
about the primes where the curve is singular. The arithmetic data at those
primes enters the Birch Swinnerton-Dyer conjecture by means of the Tamagawa
Numbers.
Again let
$$ E_n : y^2 = x^3 - n^2 x. $$
If $p \not| \,\, \Delta E = 4n^2$ then the reduction $\tilde{E} : y^2 \equiv x^3 - n^2 x \,
(p)$ is
also an elliptic curve. We have

\begin{prop}[Reduction Modulo p]
  The reduction map $E(\QQ) \rightarrow \tilde{E}(\FF_p)$ is a group homomorphism.
\end{prop}
\begin{proof}
  \cite[See][Chapter IV, pages 121-123]{rational}.
\end{proof}

If $p | \Delta E$ then $\tilde{E}(\FF_p)$ is not a group, but
\[
  \tilde{E}(\FF_p)_{ns} := \{P \in \tilde{E}(\FF_p) \,|\, P \text{ is nonsingular}\}
  \]
is still a group.

Looking at $E$ in the completion $\QQ_p$ we still have a map
\[\EQp \rightarrow \EFp\].
If we let $\EQp_0$ be the preimage of $\EFp_{ns}$ we get a group homomorphism
\[ \EQp_0 \rightarrow \EFp_{ns} \].

\begin{defn}
  The Tamagawa Number of $E$ at $p$ is the number
  $$c_p = \big| \frac{\EQp}{\EQp_0}
  \big|.$$
  If $p \not| \,\, \Delta$ then $c_p = 1$.
\end{defn}  
  
\subsection{The BSD Conjecture and Tunnell's Theorem}

During the 1960s, Bryan Birch and Peter Swinnerton-Dyer formulated, based
on numerical evidence, an influential conjecture relating the Hasse-Weil
L-function to the arithmetic invariants of $E$.

\begin{conj}[Birch Swinnerton-Dyer Conjecture]
  The Hasse-Weil L-function $L(E,s)$ can be written as
  \[L(E,s) = C (s-1)^r + O((s-1)^2)\] where
  \[C = \frac{|\Sha(E)| \, \Omega_E R_E}{|E(\QQ)_{\text{tors}}|^2} \prod_p c_p\]
\end{conj}

In particular, near 1, it predicts the value of the integer
\[0^r \cdot | \Sha(E)| \prod_p c_p .\]

The BSD conjecture is particularly powerful to gather information about
parametrised families of elliptic curves, since we can then form a parametric
family of L-functions and hope that by analysing those we can recover useful
arithmetic information.
In the case of our family $E_n$, the following remarkable result, proven using
the machinery of modular forms, shows that
$L(E_n,s)$ takes a particularly simple form.

\begin{thm}[Tunnell's Theorem]
  \begin{equation} \label{eq:tunnell}
    L(E_n,1) = \begin{cases}
      \frac{1}{2}\Omega_{E_n}a^2_n & \text{ if $n$ is odd} \\
      \Omega_{E_n}a'^2_{n/2} & \text{ if $n$ is even}
    \end{cases}
  \end{equation}

  where the $a_i, a'_i$ are coefficients of the fourier expansions of the
  functions $f, f'$ given by
  \[f(z) = \sum\limits_{m=-\infty}^\infty a_m q^m
    = (\Theta(z) - \Theta(4z)) (\Theta(32z)-\frac{1}{2} \Theta(8z)) \Theta(2z)\]
  and
  \[f'(z) = \sum\limits_{m=-\infty}^\infty a'_m q^m =
    (\Theta(z) - \Theta(4z)) (\Theta(32z)-\frac{1}{2} \Theta(8z)) \Theta(4z)\]
  with $q = e^{2\pi i z}$.
  
  Here $\Theta(z)$ is the theta-function given by
  \[ \Theta(z) = \sum\limits_{n = \infty}^{\infty} q^{n^2} \text{ where } q =
    e^{2\pi iz}.\]
\end{thm}

\begin{proof}
  \cite[See][pages 325-328]{Tunnell}. Also \cite[pages 212-222]{modular}
  discusses the proof extensively.
\end{proof}
  
In this project we will take advantage of the fact that the coefficients of
$L(E_n,1)$ are given by relatively simple arithmetic functions, and use that
to verify the BSD conjecture for the family $E_n$.

To do this we will first compute all the invariants of $E_n$ as well as
possible, and then find a simple form for the coefficients of $f$ and $f'$ modulo
a high enough power of 2. This will enable us to predict the rank of $E_n$ as
$n$ ranges over the prime numbers.


  
\section{Galois Cohomology}

\subsection{Group Cohmology}

Let $G$ be a finite group. We define the \defin{group ring $\ZZ[G]$} to
be the ring of formal sums
\[ \ZZ[G] = \big\{ \sum\limits_{i=1}^{|G|} x_i \, | \, x_i \in \ZZ, g_i \in G \big\}\]
with identity $1 \cdot e_G$ and addition and multiplication defined by
\[\sum\limits_{i=1}^{|G|} x_ig_i + \sum\limits_{i=1}^{|G|} y_ig_i =
\sum\limits_{i=1}^{|G|} (x_i+y_i)g_i\]
and \[(\sum\limits_{i=1}^{|G|} x_ig_i) \cdot (\sum\limits_{i=1}^{|G|} y_ig_i) =
\sum\limits_{t=1}^{|G|} \sum\limits_{i+j = t} x_iy_jg_ig_j \].
We say an abelian group $A$ is a \defin{$G$-module} if it is a module
for the ring $\ZZ[G]$. This is a commutative module if $G$ is also abelian.

We define the
first and second cohomology groups of $A$ by
\[H^0 (G, A) = \text{Hom}_{\ZZ[G]}(\ZZ, A)\] and
\[H^1 (G, A) = \text{Ext}^1_{\ZZ[G]} (\ZZ, A)\]
where $\ZZ$ is considered the trivial $G$-module where $g x = x$ for every
$x \in \ZZ, g \in G$.
Let $\phi : \ZZ \rightarrow A$ be a module homomorphism. Then
\[\begin{split} \phi(1) &= \phi(g \cdot 1) \,\,\,\, \forall g \in G \\
    &= g \cdot \phi(1) , \end{split}\]
so $\phi(1) \in A^G = \{x \in A \, | \, gx = x, \forall g \in G\}$, the set
of elements at which $G$ acts trivially. Since a $\ZZ$ homomorphism is
completely determined by the image of 1, we can set
\[ H^0 (G, A) = \text{Hom}_{\ZZ[G]}(\ZZ, A) = A^G .\] 

We can also define the first cohomology group as
\[H^1(G, A) = \frac{Z^1(G,A)}{B^1(G,A)}\]
where
\[Z^1(G,A) = \{ \phi : G \rightarrow A \, | \phi(gh) = \phi(g) + g \phi(h)\}\]
\[B^1(G,A) = \{ \delta \in Z^1 \, | \, \exists a \in A \,\,\text{such that}\,\, \delta(g)
  = ga - a , \forall g \in G\}\]

\begin{prop}
  With this setup, we can take a short exact sequence of $G$-modules
  \[ 0 \rightarrow A \rightarrow B \rightarrow C \rightarrow 0\]
  and form the long exact sequence
  \[ 0 \rightarrow H^0(G, A) \rightarrow H^0(G,B) \rightarrow H^0(G,C)
    \rightarrow H^1(G, A) \rightarrow H^1(G,B) \rightarrow H^1(G,C) \]
\end{prop}

\subsection{Galois Cohomology}

We can now use the machinery of group cohomology to study number fields.
Let $L, K$ be number fields and $L : K$ a finite degree Galois
extension with Galois group $G_{L:K}$.
A Galois module $A$ is a module over $G_{L:K}$.

\begin{example}
  We can consider the Galois modules $A = L \simeq K[G], $ or $A =
  L^{\times}$ where $L^{\times}$ is the multiplicative group of $L$.
\end{example}


\begin{defn}
  The \defin{zeroth and first Galois cohomology groups} are written
        \[H^0(L:K, A) = H^0(G_{L:K}, A) = A^{G_{L:K}} \]
        \[H^1(L:K, A) = H^1(G_{L:K}, A) \]
\end{defn}

\begin{example}
  If $E$ is an elliptic curve over $K$ then $A = E(L)$ is a Galois module,
  since the addition formulas are rational over $K$.

  If $K = \QQ$, we have $H^0(\QQ,E(L)) = E(L)^{G_{L:\QQ}} = E(\QQ)$.
\end{example}

From now on we will write $H^1(\QQ, E)$ for the group $H^1(\QQ, E(\bar{\QQ}))$.kk

The following result gives an important terminating condition for the
long exact sequences:

\begin{thm}[Hilbert's Theorem 90]
  $  H^1(L : K, L^{\times}) = 0$
\end{thm}
\begin{proof}
  \cite[See][Chapter X, page 150]{cohomology}
\end{proof}

In what follows we will take $L = \bar{K}$, the algebraic closure of K and
$G_K = G_{\bar{K}:K}$ its Galois group. This extension is usually infinite, so
it will be necessary to make amendmends to the previous defintions.

\begin{defn}
  A $G_K$-module $A$ is called a continuous $G_K$-module if for all $g \in G_K, a \in A$,
  there exists a finite Galois extension $L:K$ such that $g(a)$
  depends only on the image of $g$ in $G_{L:K}$.
\end{defn}

\begin{example}
  $\bar{K}$ and $ \bar{K}^{\times}$ are continuous $G_K$-modules.
\end{example}

\begin{lemma}
  $E(\bar{K})$ is a continuous $G_K$-module.
\end{lemma}
\begin{proof}
  Note that if $P = (x,y) \in E(\bar{K})$ and $L$ the field generated
  by the coordinates of $P$, then $P \in E(L)$ and $L$ is finite. This is
  because $x, y$ satisfy an algebraic equation over $K$.
  But then $g P = P, \forall g \in G_{\bar{K} : L}$, as wanted.
\end{proof}
  
To form cohomology groups, set
\[H^1(K,A) = \frac{Z^1_{\text{cts}}(K,A)}{B^1(K,A)}\]
where
\[Z^1_{\text{cts}} = \{\phi : G_K \rightarrow A \, | \, \exists L : K
  \text{ such that } \phi(g) \text{ depends only on } g \text{ modulo } L\}.\]

If we have a short exact sequence of continuous $G_K$-modules
\[ 0 \ra A \ra B \ra C \ra 0\]
this gives a long exact sequence of Galois cohomology groups
\[0 \ra A^{G_K} \ra B^{G_K} \ra C^{G_K} \ra H^1(K,A) \ra H^1(K,B) \ra H^1(K,C).\]


\subsection{Applications to Elliptic Curves} \label{sec:applications}
A crucial step in the proof of the Mordell-Weil theorem, enough to show that
the rank of an elliptic curve over $\QQ$ is finite, is the study of the size
of the quotient $E(\QQ) / 2 E(\QQ)$. In curves with at least one 2-torsion
point, this can be done with the help of an isogeny.

Let $$E : y^2 = x^3 + ax^2 + bx$$ be an elliptic curve over $\QQ$. We know $E$
has the rational points $\OO, T = (0,0)$.

We also consider the curve
\[ \bar{E} : y^2 = x^3 + \bar{a}x^2 + \bar{b}x \]
where $\bar{a} = -2a$ and $\bar{b} = a^2 - 4b$. Repeating this process yields
the curve
\[ \bar{\bar{E}} : y^2 = x^3 +  4ax^2 + 16bx\]
which is birationally equivalent to $E$ by the transformation $y \mapsto 8y,
x \mapsto 4x$.

\begin{prop}
  Let $E, \bar{E}$ be as above. The maps $\phi : E \rightarrow \bar{E}$ and
  $\psi : \bar{E} \rightarrow E$ defined by
  \[\phi(P) =
    \begin{cases}
      (\frac{y^2}{x^2}, \frac{y(x^2-b)}{x^2}), & \text{ if } P \neq \OO, T \\
      \bar{\OO}, & \text{ if } P = \OO \text{ or } P = T
    \end{cases}\]
  and
  \[\psi(P) =
    \begin{cases}
      (\frac{\bar{y}^2}{4\bar{x}^2}, \frac{\bar{y}(\bar{x}^2-\bar{b})}{8\bar{x}^2}),
      & \text{ if } P \neq \bar{\OO}, \bar{T} \\
      \bar{\OO}, & \text{ if } \bar{P} = \bar{\OO} \text{ or } \bar{P} = \bar{T}
    \end{cases}\]
  are elliptic curve isogenies, $\text{Ker}(\phi) = \{\OO, T\}$ and
  \[ \psi \circ \phi (P) = 2P , \text{ for all points } P \in E.\]
\end{prop}
\begin{proof}
  See \cite[See][Chapter 4, page 79]{rational}.
\end{proof}

\begin{lemma} \label{lemma:rankformula}
  If $E / \psi(\bar{E}))$ and $\bar{E} / \phi(E)$ are finite, then so is $E/2E$.
  In fact, the rank $r$ of $E$ satisfies
  \[2^r = \frac{\#E/\psi(\Eb) \cdot \#\Eb/\phi(E)}{4}\]
\end{lemma}
\begin{proof}
  See \ref{rational}.
\end{proof}

We are thus led to consider the quotient $E(\QQ) / \psi(\bar{E}(\QQ))$ (the
other one can be treated identically).

By the proposition, we have a short exact sequence of $G_{\QQ}$-modules
\[ 0 \rightarrow \{\OO, T\} \rightarrow \bar{E}(\bar{\QQ}) \xrightarrow[]{\psi}
  E(\bar{\QQ})
  \rightarrow 0\]
and $\{\OO, T\} \simeq \ZZ/2\ZZ$.

Taking Galois cohomology we get the long exact sequence

\[ 0 \rightarrow \ZZ / 2\ZZ \rightarrow \bar{E}(\QQ) \xrightarrow[]{\psi} E(\QQ)
  \rightarrow H^1(\QQ, \ZZ/2\ZZ) \rightarrow H^1(\QQ, \bar{E}) \xrightarrow{H^1(\psi)}
  H^1(\QQ, E)\]
This in turn gives us the short exact sequence
\[ 0 \rightarrow \frac{E(\QQ)}{\psi(\bar{E} (\QQ))} \rightarrow
  H^1(\QQ, \ZZ / 2\ZZ) \rightarrow H^1(\QQ, \bar{E}(\QQ))[\psi] \rightarrow 0\]

where $H^1(\QQ,\bar{E}(\QQ))[\psi] = (H^1(\psi))^{-1}.$ \vspace{3mm}

To get the order of $E(\QQ) / \psi(\bar{E}(\QQ))$ we need to investigate the group
$H^1(\QQ, \ZZ/2\ZZ)$.

\begin{prop}
  There exists a canonical isomorphism so that
  \[H^1(\QQ, \ZZ/2\ZZ) \simeq \QQ^{\times} / (\QQ^{\times})^2\]
\end{prop}
\begin{proof}
Consider the exact sequence of Galois modules
\[0 \rightarrow \mu_2 \rightarrow \bar{\QQ}^{\times} \xrightarrow[]{2} \bar{\QQ}^{\times} \ra
  \, 0 \]
where $\mu_2 = \text{Gal} (\QQ(i) : \QQ) \simeq \ZZ / 2\ZZ.$

Taking cohomology gives the long exact sequence
\[ 0 \ra \mu_2 \ra \bar{\QQ}^{\times} \xrightarrow[]{2} \bar{\QQ}^{\times} \ra H^1(\QQ, \ZZ/2\ZZ)
  \ra H^1(\QQ, \bar{\QQ}^{\times})\]
and $H^1(\QQ, \bar{\QQ}^{\times}) \simeq 0$ by Hilbert's Theorem 90. Thus,
\[H^1(\QQ, \ZZ/2\ZZ) \simeq \QQ^{\times}/(\QQ^{\times})^2. \]
\end{proof}

\subsection{The Selmer and Tate-Shafarevich Groups}

In the effort to understand $E/2E$, we were led to consider the quotient
$E(\QQ) / \psi(E(\QQ))$. Our application of Galois cohomology showed that this
group is a subsgroup of the multiplicative group of rationals modulo squares
$\QQ^{\times} / (\QQ^{\times})^2$. This group, in turn, can be studied by means
of local method.

\begin{defn}
  A \defin{place} $\nu$ is either a prime number $p$ or $\infty$. $\QQ_\nu$ then is
  either the field of p-adic numbers if $\nu = p$ or $\RR$ if $\nu = \infty$.
\end{defn}

\begin{defn}
  If $\phi: A \rightarrow B$ is a homomorphism, then write $A[\phi]$ for the
  kernel of $\phi$.
\end{defn}

There is then a natural inclusion $\QQ \ra \prod\limits_\nu \QQ_\nu,
x \mapsto (x, x, \dots)$ since $\QQ \subset \QQ_{\nu}$ for each $\nu$.

\begin{tikzcd}
  0 \arrow[r] &
  E(\QQ)/\psi(E(\QQ)) \arrow[r] &
  \QQ^{\times}/(\QQ^{\times})^2 \arrow[r] \arrow[d] &
  H^1(\QQ, \bar{E})[\psi] \arrow[r] \arrow[d] &
  0 \\
  0 \arrow[r] &
  \prod\limits_{\nu} E(\QQ_\nu) / \psi(E(\QQ_\nu)) \arrow[r] &
  \prod\limits_\nu(\QQ_\nu^{\times}/(\QQ_\nu)^{\times})^2 \arrow[r] &
  \prod\limits_\nu H^1(\QQ_\nu, \bar{E})[\psi] \arrow[r] &
  0
\end{tikzcd}

We are now ready to make the following definition

\begin{defn}[Selmer Group]
  The Selmer group of the elliptic curve $E$ is defined
  $$\Sel = \text{Ker}\big((\QQ^{\times}/(\QQ^{\times})^2 \ra
  \prod\limits_\nu H^1(\QQ_\nu, \bar{E})[\psi]\big)$$
\end{defn}

This group is important in rank computations primarily because it is effectively
computable. To understand $\EQ/\psi(\Eb(\QQ))$ fully, we also need

\begin{defn}[Tate-Shafarevich Group] \label{defn:sha}
  The Tate-Shafarevich group of the elliptic curve $E$ is defined
  $$ \TS(\bar{E} / \QQ) = \text{Ker} \big((H^1(\QQ, \bar{E}) \ra \prod\limits_\nu H^1 (\QQ_\nu,
  \bar{E})[\psi]\big) $$
\end{defn}

Finally, we have the important short exact sequence

\[ 0 \ra \frac{E(\QQ)}{\psi(\bar{E}(\QQ))} \ra \Sel \ra \TS(\bar{E}/\QQ)[\psi] \ra 0\]

If $\TS(\Eb/\QQ)[\psi] \simeq 0$, $Sel \simeq \EQ/\psi(\Eb(\QQ))$. By playing
the same game with the isogeneous curve $\Eb$, if $\TS(E/\QQ)[\phi] \simeq 0$,
we also know $\Eb(\QQ)/\phi(\EQ)$ and can use \ref{lemma:rankformula} to compute
the rank of $E$. 

\section{Calculations}

\subsection{Introduction}

The Birch Swinnerton-Dyer conjecture states that the Hasse-Weil L-function
$L(E,s)$ of the elliptic curve $E$ can be written as
\[L(E,s) = C (s-1)^r + O((s-1)^2)\] where
\[C = \frac{|\text{Sha}(E)| \, \Omega_E R_E}{|E(\QQ)_{\text{tors}}|^2} \prod_p c_p\]
In particular, near 1, it predicts the value of
\[0^r \cdot |\text{Sha}(E)| \prod_p c_p \in \ZZ\].

From Tunnell's theorem we know that $L(E_n,s)$ has a particularly simple
form in terms of Theta functions. In this subsection we'll show that they
can be used to predict whether $0^r |\text{Sha}(E)$ is even or odd, and that
this agrees with what's the BSD conjecture.


\subsection{Coefficients of Theta Series}
Tunnell's theorem requires us to consider the quantities 
\[  A_n = \#\{(x,y,z) | n = 2x^2 + y^2 + 32z^2\} \]
\[  B_n = \#\{(x,y,z) | n = 2x^2 + y^2 + 8z^2\} \]
\[  C_n = \#\{(x,y,z) | n = 8x^2 + 2y^2 + 64z^2\} \]
\[  D_n = \#\{(x,y,z) | n = 8x^2 + 2y^2 + 16z^2\} \]

depending on whether $n$ is odd or even. For our calculations we'll
need to compute $A_n - \frac{1}{2}B_n$ and $C_n - \frac{1}{2}D_n$ modulo 4.

\begin{thm}
  $2A_p - B_p \equiv \begin{cases} 0, & p \equiv 1, 5, 7 \, (8) \\
    2, & p \equiv 3 \, (8) \end{cases}
  \, (4)$
\end{thm}
\begin{proof}
  In computing $A_P$ and $B_P$ we only need to consider solutions
  where at least 1 of $x, y, z$ is 0, since if neither is 0, all of $\pm x, \pm y, \pm z$
  are solutions, and hence they together don't make a contribution to the total
  number of solutions modulo 8. Furthermore, since we only need to consider $A_n$
  modulo 4, we can also ignore solutions where exactly 1 of $x, y z$ are 0. But
  there are no remaining solutions since p is prime and if 2 of $x, y, z$ are 0 we
  arrive at the contradiction $x^2 = p$, so we have $A_p \equiv 0 \, (4)$.

  With this we're left to consider $\frac{1}{2}B_p \, (4)$, or $B_p \, (8)$. Again we can
  ignore solutions where all of $x, y, z$ are 0, so we have, by inclusion-exclusion

  \( A_p \equiv \#\{ 2x^2 + y^2 = p\} + \#\{y^2 + 8z^2 = p\} + \#\{2x^2 + 8z^2 = p\} \\
  \equiv \#\{ 2x^2 + y^2 = p\} + \#\{y^2 + 8z^2 = p\} \, (8)\)
  since p is odd. To find the remaining quantities we need some algebraic number
  theory.

\begin{lemma}
  For $p$ odd, $\#\{2x^2 + y^2 = p\} = \begin{cases} 4,
    p \equiv 1 \, (8) \\
    1, p \equiv 3 \, (8) \\
    0, else \end{cases}$
\end{lemma}
\begin{proof} We must have $2x^2 + y^2 \equiv 0 \, (p)$, so we only have solutions if
$(\frac{-2}{p}) = 1$, or equivalently if $p \equiv 1, 3 \, (8)$. Since
$\ZZ[\sqrt{2}]$ is a PID, if $p \equiv 1 \, (8), p = \pi \bar{\pi}$ for some
prime $\pi = x + y \sqrt{2}$ where $x^2 + 2y^2 = p$, so $x$ is odd and $x^2
\equiv 1 \, (8)$
If $y$ is also odd, $1 + 2y^2 \equiv 1 \, (8) \implies y^2 \equiv 1 \, (8)$, and
we have $x^2 + 2y^2 \equiv 3 \, (8)$, a contradiction.
If $y = 2k$ is even, we have $x^2 + 2y^2 = x^2 + 8y^3$ 
\end{proof}

To conclude the proof of the theorem, we can deduce from the lemma that
\begin{itemize}
  \item If $p \equiv 1 \, (8), A_p \equiv 4+4 \equiv 0 \, (8)$.
  \item If $p \equiv 3 \, (8), A_p \equiv 4 + 0 \equiv 4 \, (8)$.
  \item If $p \equiv 5, 7 \, (8), A_p \equiv 0 + 0 \equiv 0 \, (8)$.
\end{itemize} 
Looking at $A_p - \frac{1}{2} B_p$ modulo 4 gives the result.
\end{proof}
\subsection{Selmer Group}

Now we proceed in the other direction, and examine the expected rank of the
curve by studying its arithmetic invariants. We'll first compute the Selmer
group of $E_p$ and $\Eb_p$.

\begin{prop}
  $\Sel(E_p)$ is a subgroup of the multiplicative group
  $ \{ \pm 1, \pm p \}$, and $d_1 \in \Sel$ if and only if the equation
  \begin{equation} \label{eq:hom}
    N^2 = d_1 M^4 + \frac{p^2}{d_1}e^4
  \end{equation}
  has a solution in $\QQ_p$ for all $p$.
  Similarly, $\Sel(\Eb_p)$ is a subgroup of $\{\pm 1, \pm 2, \pm p, \pm 2p\}$
  consisting of elements $d_1$ at which \ref{eq:hom} has a p-adic solution for
  all prime numbers.
\end{prop}
Proof: See \cite{rational}

We will compute $\Sel$ by looking at the equation modulo different primes.
There will be 3 cases to consider: 2, $p$, and any prime $l \neq p$.

\subsubsection{l-adic case}

In order to work out if equation \ref{eq:hom} has a solution in $\QQ_p$,
we will first show it has a solution 
in $\ZZ/p\ZZ$, and then lift this solution to $\QQ_p$ using:

\begin{prop}[Hensel's Lemma]
  Let $f \in \ZZ_p[X_1,\dots,X_m], x = (x_i) \in (\ZZ_p)^m, n, k \in \ZZ$ and
  $j$ an integer such that $0 \leq j \leq m$. Suppose that $0 < 2k < n$ and that
  \[ f(x) \equiv 0 \, (p^n) \text { and } \nu_p
    \big(\frac{\partial f}{\partial X_j} (x) \big) = k.\]
  Then there exists a zero $y$ of $f$ in $(\ZZ_p)^m$ which is congruent to $x$
  modulo $p^{n-k}$.
\end{prop}

To show existence of local solutions, we will need the following results: 

\begin{prop} 
  The number of solutions $N_p$ to the equation
  \[a_1x_1^{l_1} + \cdots a_rx_r^{l_r} \equiv 0 \, (p)\]
  satisfies
  \begin{equation} 
    |N - p^{r-1}| \leq M(p-1)p^{(r/2)-1}
  \end{equation} 
  where $M$ is the number of $r$-tuples
  of characters \linebreak
  $\chi_1, \dots , \chi_r$
  where $\chi_i^{l_i} = \varepsilon, \chi_i \neq \varepsilon$ for
  $i = 1, \dots, r$ and $\chi_1 \chi_2 \cdots \chi_r = \varepsilon$.
\end{prop} \label{prop:countsols}
Proof: \cite[See][]{classical}.

\begin{prop}[Chevalley Warning Theorem]
  Let $f \in K[X_1, \dots , X_n]$ be a homogeneous polynomial
  in $n$ variables such that $deg f \leq n$. Then $f$ has a nontrivial 0.
\end{prop}

\begin{proof}
  \cite[See][Chapter 1, page 5]{Serre}
\end{proof}

We can apply these results to show that equation \ref{eq:hom} has solutions
in $\ZZ/p\ZZ$.

\begin{lemma}
  If $l \neq 2, p$ is prime, then \ref{eq:hom} has a solution in $\QQ_l$.
\end{lemma}
Proof: We are looking at the number of solutions to
\[ N^2 - d_1M^4 - \frac{p^2}{d_1}e^4 \equiv 0 \, (l)\]
where $d_1 = \pm 1, \pm 2, \pm p, \pm 2p$. Since none of $1, d_1$ and
$\frac{p^2}{d_1}$ are divisible by $l$, we may apply
proposition \ref{prop:countsols}.

If $l \equiv 1 \, (4)$, there are 2 characters modulo
 $l$ of order dividing 4. The primitive quartic
characters $\chi^{\pm 1}$ and the quadratic character $\chi^2$ where $\chi^2(n) =
(\frac{n}{p})$. Thus, $M$ is the number of tuples $\chi, \chi^a, \chi^b,
a,b = \pm 1, 2$ where $\chi \chi^a \chi^b = \varepsilon$, so $M = 2$. Plugin
into the bound given by proposition \ref{eq:countsols} we get
\[|N_l - l^{r-1} | \leq 2(l-1) l^{(3/2)-1}.\]
The equation \ref{eq:hom} will have a nontrivial solution if $|N-p^2| \leq
p^2-2$, but
\[2(p-1)p^{1/2} \leq p^2 - 2\]
is satisfied for all $p \geq 3$.
If $l \equiv 3 \, (4)$, the values taken by $x^4$ are the same as those taken
by $x^2$, so $N_l$ is also the number of solutions to
\[N^2 - d_1M^2 - \frac{p^2}{d_1} e^2 \equiv 0 \, (l),\]
which has a nontrivial solution by the Chevalley Warning theorem. \qedhere

\begin{thm}[Existence of l-adic Solutions]
  For $a, b \not\equiv 0 \, (l)$, there is always a triple
  $x, y, z \in \QQ_p^{\times}$ such that $x^2 = ay^4 + bz^4$.
\end{thm}
Proof: Let $f(x,y,z) = x^2 - ay^4 - bz^4$. Then $f$ has a nontrivial root
$(X,Y,Z)$ in $\ZZ/p\ZZ$ by the previous results, so we may assume for instance
that $X \not\equiv 0 \, (p)$.
Since $l$ is odd, neither partial derivative increases the $l$-adic valuation
of $f$, so, for instance, $\nu_l(\frac{\partial f(X,Y,Z)}{\partial x}) =
\nu_l(2X) = 0$ and $f(X,Y,Z) \equiv 0 \, (p)$, so $\nu_p(f(X,Y,Z)) \geq 1$.
Thus, Hensel's lemma applies and we get a solution $\bar{X}, \bar{Y}, \bar{Z}
\in \QQ_p$. \qedhere

\subsubsection{p-adic case}
Now there are a finite number of equations to check. We will proceed case by
case. Remeber we're trying to find solutions in $\QQ_p$ to the equations
\begin{equation} \label{eq:homcurve}
  N^2 = d_1 M^4 - \frac{p^2}{d_1}e^4
\end{equation}
and
\begin{equation} \label{eq:isocurve}
  N^2 = d_1 M^4 + \frac{4p^2}{d_1}e^4
\end{equation}
corresponding to the curves $E_p$ and $\Eb_p$.

\begin{thm}
  The equations \ref{eq:homcurve} have a nontrivial solution in $\QQ_p$. In fact,
  they have solutions in $\QQ$.
\end{thm}
Proof:
\begin{itemize}
\item If $d_1 = -1$, we have the rational solution $(p, 0, 1)$.
\item If $d_1 = p$, we have the rational solution $(0, 1, 1)$.
\item If $d_1 = -p$, we have the rational solution $(0, 1, 1)$.
\end{itemize}
\qedhere

\begin{thm}
  The equations \ref{eq:isocurve} have a nontrivial solution in $\QQ_p$ subject
  to the following conditions: 

  $\begin{tabu}{l*{2}{c}r}
    d_1 & \text{condition} \\
    \hline
    1 & \cmark \\
    2 & p \equiv \pm 1 \, (8) \\
    p & p \equiv 1 \, (4) \\
    2p & p \equiv 1 \, (8) \\
    -1 & p \equiv 1 \, (4) \\
    -2 & p \not\equiv 5 \, (8) \\
    -p & p \equiv 1 \, (4) \\
    -2p & p \equiv 1 \, (8)
  \end{tabu}$
\end{thm}
\begin{proof} Proceeding case by case:
  \begin{itemize}
  \item If $d_1 = -1, N^2 \equiv -M^4 \, (p^2)$, so if $M \not\equiv 0 \, (p^2),
    -1 \equiv (\frac{N}{M^2})^2 \, (p)$, which has a solution if and only if
    $p \equiv 1 \, (4)$. If $M \equiv 0 \, (p^2), M = p^2M'$, so $N$ is also
    divisble by $p$, and we don't have a nontrivial solution in $\ZZ/p\ZZ$.
  \item If $d_1 = 2, N^2 \equiv 2M^4 \, (p^2)$, and by a similar argument we must
    have $(\frac{2}{p}) = 1$, so we have a solution if and only if $p \equiv \pm 1
    \, (8)$.
  \item If $d_1 = p, N^2 \equiv 0 \, (p)$, so $N = pN'$ and $pN'^2 = M^4 + 4e^4$.
    Since $M, e$ cannot be divisible by $p$, we must have
    $-4 \equiv (\frac{M}{e})^4 \,\, (p)$. If $p \equiv 3 \, (4)$, this is
    equivalent to $-4 \equiv X^2 \, (p)$, which is a contradiction.
    Conversely, if $p \equiv 1 \, (8),$ 1 is a 4th power and $ (\frac{2}{p}) = 1$,
    so $-4 = (\sqrt{2} \cdot \sqrt{-1})^4$. If $p \equiv 5 \, (8)$, let $\chi$ be
    a primitive quartic character. Then $\chi(-4) = \chi(-1)\chi(2)^2
    = \chi(-1) (\frac{2}{p}) = (-1) \cdot (-1) = 1$, since -1 is not a fourth
    power. Thus, we have a solution if and only if $p \equiv 1 \, (4)$.
  \item If $d_1 = 2p, N^2 \equiv 0 \, (p)$, so $N = pN'$ and $pN'^2 = 2M^4 +
    2e^4$. Again, since we have$M, e \not\equiv 0 \, (p), -1 \equiv (\frac{M}{e})^4 \,
    (p)$, so we must have $p \equiv 1 \, (8)$.
  \item If $d_1 = -2$, a similar argument shows we must have $(\frac{-2}{p})=1$,
    so $p \not\equiv 5 \, (8)$.
  \item If $d_1 = -p$, again we have solutions if and only if -4 is a 4th power,
    and then $p \equiv 1 \, (4)$. 
  \item If $d_2 = -2p$, again we have solutions if and only if $p \equiv 1 \, (8)$.
  \end{itemize} \qedhere
\end{proof}

\subsubsection{2-adic case}
Finally, we consider the 2-adic solutions. Remembering \ref{eq:homcurve} has
rational solutions, we only need to consider the isogeneous curve.

\begin{thm}
  The equations \ref{eq:isocurve} have a nontrivial solution in $\QQ_2$ subject
  to the following conditions:
  
  $\begin{tabu}{l*{2}{c}r}
    d_1 & \text{condition} \\
    \hline
    1 & \cmark \\
    2 & \cmark \\
    p & p \equiv 1 \, (4) \\
    2p & p \equiv 1 \, (4) \\
    -1 & \xmark \\
    -2 & \xmark \\
    -p & p \equiv 3 \, (4) \\
    -2p & p \equiv 3 \, (4)
  \end{tabu} $
\end{thm}
\begin{proof}
  Note that $x^4 \equiv 1 \, (16)$ if $x$ is odd and 0 if it's even. Also
  note that the squares modulo 16 are 0, 1, 4 and 9. Since
  $M, e$ cannot both be even, we can check the following cases:
  \begin{itemize}
    \item If $d_1 = 2$, $N^2 \equiv \begin{cases} 2 + 2p^2 \\ 2 \\
        2p^2 \end{cases} \, (16)$. \newline In particular, since $p^2 = 1$ or 9,
      we can take the option
      $2 + 2p^2 \equiv 4 \equiv 2^2 \, (16)$.
    \item If $d_1 = p$, $N^2 \equiv \begin{cases} p + 4p \\ p \\
        4p \end{cases} \, (16)$. \newline If $p \equiv 1 \, (4), 4p \equiv 1 \, (16)$.
      The first equation has no solution since $5^{-1} \equiv 13 \not\equiv N^2
      \, (16)$.
    \item If $d_1 = 2p$, $N^2 \equiv \begin{cases} 2p +2p \\ 2p \\
        2p \end{cases} \, (16)$. \newline We have a solution if $p \equiv 1
      \, (4)$.
    \item If $d_1 = -1$, $N^2 \equiv \begin{cases} -1 -4 p^2 \\ -1 \\
        -4p^2 \end{cases} \, (16)$. \newline There are no solutions, since $p^2 = 1, 9$.
    \item If $d_1 = -2$, $N^2 \equiv \begin{cases} -2 - 2p^2 \\ -2\\
        -2p^2 \end{cases} \, (16)$. \newline There are also no solutions.
    \item If $d_1 = -p$, $N^2 \equiv \begin{cases} -p -4p \\ -p \\
        -4p \end{cases} \, (16)$. \newline The first 2 equations have no solutions, 
      since $-5^{-1} \equiv 3$ and $-1^{-1} \equiv -1$, and neither is a square.
      If $N^2 = 4 \equiv -4p \, (16), p \equiv -1 \, (4)$.
    \item If $d_1 = -2p$, $N^2 \equiv \begin{cases} -2p -2p \\ -2p \\
        -2p \end{cases} \, (16)$. If $N^2 \equiv 0 \, (16)$, there are no
      solutions since p is odd. If $N^2 \equiv 4 \, (16), p \equiv -1 \, (4)$.
  \end{itemize}
\end{proof}

We can now put everything together to work out the Selmer rank of $E_p$.

$  \begin{tabu}{l*{5}{c}r}
    d_1 & \text{l-adic} & \text{p-adic} & \text{2-adic} & \RR & \Sel \\
    \hline
    1 & \cmark & \cmark & \cmark & \cmark & \cmark \\
    2 & \cmark & p \equiv \pm 1 \, (8) & \cmark & \cmark & p \equiv \pm 1 \, (8) \\
    p & \cmark & p \equiv 1 \, (4) & p \equiv 1 \, (4) & \cmark & p \equiv 1 \, (4) \\
    2p & \cmark & p \equiv 1 \, (8) & p \equiv 1 \, (4) & \cmark & p \equiv 1 \, (8)\\
    -1 & \cmark & p \equiv 1 \, (4) & \xmark & \xmark & \xmark \\
    -2 & \cmark & p \not\equiv 5 \, (8) & \xmark & \xmark & \xmark \\
    -p & \cmark & p \equiv 1 \, (4) & p \equiv 3 \, (4) & \xmark & \xmark \\
    -2p & \cmark & p \equiv 1 \, (8) & p \equiv 3 \, (4) & \xmark & \xmark 
  \end{tabu}$

\begin{cor}
  The Selmer rank of $E$ is equal to 2 if $p \equiv 1 \, (8), 1$ if $p \equiv
  3, 7 \, (8)$, and 0 if $p \equiv 5 \, (8)$.
\end{cor}
\begin{proof}
  Let $r$ be the Selmer rank, then
  \[2^r = \frac{\#\Sel(E) \cdot \#\Sel(\Eb)}{4} = \#\Sel(\Eb) =
    \begin{cases}
      4 & \text { if } p \equiv 1 \, (8) \\
      2 & \text{ if } p \equiv 3, 7 \, (8) \\
      1 & \text { otherwise}
    \end{cases}\]
\end{proof}

\subsection{Tamagawa Numbers}


\begin{thm}
  The Tamagawa Number of $E_p$ at $p$ is 4.
\end{thm}
Proof: Working in $\QQ_p$, if $x \equiv 0$ then $y^2 = x^3 - p^2 x \equiv 0 \,
(p^2) \implies y \equiv 0 \, (p)$, so $(x, y) \equiv (0,0) \, (p)$.
If $x$ has nonpositive valuation, $x = p^{-n} u, u \in \ZZ_p^{\times}$ and $y^2 =
p^{-3n}u^3 - p^{2-n} u$, so $n$ must be even. 

Let $n = 2m$. Then $p^{6m} y^2 = u^3 - p^{2+2m}u \implies y = p^{-3m}v, v \in
\ZZ_p^{\times}$. Since $u, v \not\equiv 0 
\, (p)$, this has solutions if and only if $(\frac{u}{p}) = 1$, so $x$ is the
reciprocal of a square in $\ZZ_p$.

Setting $x = \frac {1}{r^2}, r^2 \in \ZZ_p$ we get $y^2 = \frac{1}{r^6} -
\frac{p^2}{r^2} = \frac{1 - p^2r^4}{r^6}$, so we have a bijection
\[\ZZ_p \rightarrow \EQp_0\]
\[r \mapsto (\frac{1}{r^2}, \frac{\sqrt{1-p^2r^4}}{r^3})\]
\[0 \mapsto \mathcal O\]
To prove the theorem we will need the following lemma:

\begin{lemma}
  The torsion points $\OO, (0,0), (p,0), (-p,0)$ are a complete set of
  representatives for $\frac{\EQp}{\EQp_0}$.
\end{lemma}
Proof: Let $P(r) = (\frac{1}{r^2}, \frac{\sqrt{1-p^2r^4}}{r^3}) = (x,y)$. We will
compute $S := P(r) + Q$ for each torsion point $Q$.
\begin{itemize}
  \item $Q = \OO$: $P(r) + \OO = P(r)$.
  \item $Q = (0,0)$:
    We have $\lambda = \frac{y}{x}$, so
    \begin{equation*}
      \begin{split}
        x(S)
        &= \lambda^2 - x - 0 =
        \frac{\frac{1 - p^2r^4}{r^6}}{\frac{1}{r^4}} - \frac{1}{r^2} \\
        &= -p^2r^2
      \end{split}
    \end{equation*}
  \item $Q = (p, 0)$: We have $\lambda = \frac{y}{x-p}$, so
    \begin{equation*}
      \begin{split}
        x(S)
        &= \lambda^2 - x - p = 
        \frac{\frac{1 - p^2r^4}{r^6}}{\frac{(1-pr^2)^2}{r^4}} -
        \frac{1}{r^2} - p \\
        &= \frac{1-p^2r^4}{r^2(1-pr^2)^2} - \frac{1+pr^2}{r^2} \\
        &= -p + \frac{2p}{1-pr^2}
      \end{split}
    \end{equation*}
  \item $Q = (-p, 0)$: We have $\lambda = \frac{y}{x+p}$, so
    \begin{equation*}
      \begin{split}
        x(S)
        &= \lambda^2 - x + p = 
        \frac{\frac{1 - p^2r^4}{r^6}}{\frac{(1+pr^2)^2}{r^4}} -
        \frac{1}{r^2} + p \\
        &= p - \frac{2p}{1+pr^2}
      \end{split}
    \end{equation*}
\end{itemize}

Now let $S = (x,y)$.
If $x \equiv 0 \, (p^2)$, $x = p^2 t, t \in \ZZ_p$ we want to show that $x = p^2r^2, r \in \ZZ_p$, so that $S$ lies
in the coset $P(r) + (0,0)$. Note that
\begin{equation*}
  \begin{split}
    y^2 &= x^3 - p^2x \\
    &= x(x+p)(x-p) \\
    x &= \frac{y^2}{(x+p)(x-p)} \\
    &= \frac{y^2}{p^2(1+tp)(1-tp)}
  \end{split}
\end{equation*}
and $1 \pm tp$ are squares in $\QQ_p$, so indeed $x$ is a square and we can set
$r = \sqrt{\frac{x}{p^2}}$.

If $x \equiv p \, (p^2), x = p + p^2t$ we want to solve
\[p - \frac{2p}{1+pr^2} = x = p + p^2t \] or
\[ \frac{-2}{1+pr^2} = pt\]
to show that $S$ lies in the coset $P(r)+(-p,0)$.
\begin{equation*}
  \begin{split}
    1+pr^2 &= \frac{-2}{pt} \\
    r^2 &= \frac{-2-pt}{p^2t}
  \end{split}
\end{equation*}.
Now
\[ x = \frac{y^2}{(x+p)(x-p)} = \frac{y^2}{(p^2t+2p)(p^2t)}
= \frac{y^2}{p^2t(2+pt)p}\]
\[ \implies x = \frac{y^2}{-r^2 p}\]
\[ \implies r^2 = \frac{y^2}{-px} = \frac{y^2}{p^2(-1-pt)}\]


If $x \equiv -p \, (p^2), x = -p + p^2t$ and we solve
\[ -p + \frac{2p}{1-pr^2} = x = -p + p^2t\] or
\[ \frac{2}{1-pr^2} = pt\]
\[ \implies r^2 = \frac{pt - 2}{p^2t}\]

Now
\[ x = \frac{y^2}{(x+p)(x-p)} = \frac{y^2}{(p^2t-2p)(p^2t)}
= \frac{y^2}{p^2t(pt-2)p}\]
\[\implies x = \frac{y^2}{r^2p}\]
\[ r^2 = \frac{y^2}{xp} = \frac{y^2}{p^2(1+pt)}\]
which is a square, so again we can solve for $r$.

\begin{thm}
  The Tamagawa Number of $E_P$ at 2 is 2.
\end{thm}
Proof: By checking all possibilities, we know that
\[\EFt = \{\OO, (0,0), (1,0)\},\]
with $(1,0)$ being the singular point. Let $\pi : E(\QQ_2)_0 \rightarrow
\EFt_{ns}$ be the projection map. 
or any $P = (x,y) \in E(\QQ_2)$, if $\pi(P) = (1,0)$ then $\pi(P * (p,0)) \neq (p,0)$.
If $x$ is a unit, let $(x,y) * (p,0) = (x',y')$. We want to show that $x'$ is
not a unit. Suppose it is, and let $\lambda$ be the gradient of the line through
$(x,y), (p,0), (x',y')$, so that $\lambda = \frac{y}{x-p}$.
Then $x + p + x' = \lambda^2$, so $\lambda \not\equiv 0 \, (2)$, so $\lambda$ is
a unit. We also have
\[
  \begin{split}
     y^2 &= x^3 - p^2x \\
     \implies \lambda^2(x-p)^2 &= x (x+p) (x-p) \\
     \implies u^2 &= \frac{x (x+p)}{x-p} 
   \end{split}
\]
implying $\frac{x+p}{x-p}$ is a unit, which is a contradiction, since
$x \pm p$ have different valuations.

\printbibliography

\end{document}